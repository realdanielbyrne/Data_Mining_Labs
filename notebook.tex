
% Default to the notebook output style

    


% Inherit from the specified cell style.




    
\documentclass[11pt]{article}

    
    
    \usepackage[T1]{fontenc}
    % Nicer default font (+ math font) than Computer Modern for most use cases
    \usepackage{mathpazo}

    % Basic figure setup, for now with no caption control since it's done
    % automatically by Pandoc (which extracts ![](path) syntax from Markdown).
    \usepackage{graphicx}
    % We will generate all images so they have a width \maxwidth. This means
    % that they will get their normal width if they fit onto the page, but
    % are scaled down if they would overflow the margins.
    \makeatletter
    \def\maxwidth{\ifdim\Gin@nat@width>\linewidth\linewidth
    \else\Gin@nat@width\fi}
    \makeatother
    \let\Oldincludegraphics\includegraphics
    % Set max figure width to be 80% of text width, for now hardcoded.
    \renewcommand{\includegraphics}[1]{\Oldincludegraphics[width=.8\maxwidth]{#1}}
    % Ensure that by default, figures have no caption (until we provide a
    % proper Figure object with a Caption API and a way to capture that
    % in the conversion process - todo).
    \usepackage{caption}
    \DeclareCaptionLabelFormat{nolabel}{}
    \captionsetup{labelformat=nolabel}

    \usepackage{adjustbox} % Used to constrain images to a maximum size 
    \usepackage{xcolor} % Allow colors to be defined
    \usepackage{enumerate} % Needed for markdown enumerations to work
    \usepackage{geometry} % Used to adjust the document margins
    \usepackage{amsmath} % Equations
    \usepackage{amssymb} % Equations
    \usepackage{textcomp} % defines textquotesingle
    % Hack from http://tex.stackexchange.com/a/47451/13684:
    \AtBeginDocument{%
        \def\PYZsq{\textquotesingle}% Upright quotes in Pygmentized code
    }
    \usepackage{upquote} % Upright quotes for verbatim code
    \usepackage{eurosym} % defines \euro
    \usepackage[mathletters]{ucs} % Extended unicode (utf-8) support
    \usepackage[utf8x]{inputenc} % Allow utf-8 characters in the tex document
    \usepackage{fancyvrb} % verbatim replacement that allows latex
    \usepackage{grffile} % extends the file name processing of package graphics 
                         % to support a larger range 
    % The hyperref package gives us a pdf with properly built
    % internal navigation ('pdf bookmarks' for the table of contents,
    % internal cross-reference links, web links for URLs, etc.)
    \usepackage{hyperref}
    \usepackage{longtable} % longtable support required by pandoc >1.10
    \usepackage{booktabs}  % table support for pandoc > 1.12.2
    \usepackage[inline]{enumitem} % IRkernel/repr support (it uses the enumerate* environment)
    \usepackage[normalem]{ulem} % ulem is needed to support strikethroughs (\sout)
                                % normalem makes italics be italics, not underlines
    

    
    
    % Colors for the hyperref package
    \definecolor{urlcolor}{rgb}{0,.145,.698}
    \definecolor{linkcolor}{rgb}{.71,0.21,0.01}
    \definecolor{citecolor}{rgb}{.12,.54,.11}

    % ANSI colors
    \definecolor{ansi-black}{HTML}{3E424D}
    \definecolor{ansi-black-intense}{HTML}{282C36}
    \definecolor{ansi-red}{HTML}{E75C58}
    \definecolor{ansi-red-intense}{HTML}{B22B31}
    \definecolor{ansi-green}{HTML}{00A250}
    \definecolor{ansi-green-intense}{HTML}{007427}
    \definecolor{ansi-yellow}{HTML}{DDB62B}
    \definecolor{ansi-yellow-intense}{HTML}{B27D12}
    \definecolor{ansi-blue}{HTML}{208FFB}
    \definecolor{ansi-blue-intense}{HTML}{0065CA}
    \definecolor{ansi-magenta}{HTML}{D160C4}
    \definecolor{ansi-magenta-intense}{HTML}{A03196}
    \definecolor{ansi-cyan}{HTML}{60C6C8}
    \definecolor{ansi-cyan-intense}{HTML}{258F8F}
    \definecolor{ansi-white}{HTML}{C5C1B4}
    \definecolor{ansi-white-intense}{HTML}{A1A6B2}

    % commands and environments needed by pandoc snippets
    % extracted from the output of `pandoc -s`
    \providecommand{\tightlist}{%
      \setlength{\itemsep}{0pt}\setlength{\parskip}{0pt}}
    \DefineVerbatimEnvironment{Highlighting}{Verbatim}{commandchars=\\\{\}}
    % Add ',fontsize=\small' for more characters per line
    \newenvironment{Shaded}{}{}
    \newcommand{\KeywordTok}[1]{\textcolor[rgb]{0.00,0.44,0.13}{\textbf{{#1}}}}
    \newcommand{\DataTypeTok}[1]{\textcolor[rgb]{0.56,0.13,0.00}{{#1}}}
    \newcommand{\DecValTok}[1]{\textcolor[rgb]{0.25,0.63,0.44}{{#1}}}
    \newcommand{\BaseNTok}[1]{\textcolor[rgb]{0.25,0.63,0.44}{{#1}}}
    \newcommand{\FloatTok}[1]{\textcolor[rgb]{0.25,0.63,0.44}{{#1}}}
    \newcommand{\CharTok}[1]{\textcolor[rgb]{0.25,0.44,0.63}{{#1}}}
    \newcommand{\StringTok}[1]{\textcolor[rgb]{0.25,0.44,0.63}{{#1}}}
    \newcommand{\CommentTok}[1]{\textcolor[rgb]{0.38,0.63,0.69}{\textit{{#1}}}}
    \newcommand{\OtherTok}[1]{\textcolor[rgb]{0.00,0.44,0.13}{{#1}}}
    \newcommand{\AlertTok}[1]{\textcolor[rgb]{1.00,0.00,0.00}{\textbf{{#1}}}}
    \newcommand{\FunctionTok}[1]{\textcolor[rgb]{0.02,0.16,0.49}{{#1}}}
    \newcommand{\RegionMarkerTok}[1]{{#1}}
    \newcommand{\ErrorTok}[1]{\textcolor[rgb]{1.00,0.00,0.00}{\textbf{{#1}}}}
    \newcommand{\NormalTok}[1]{{#1}}
    
    % Additional commands for more recent versions of Pandoc
    \newcommand{\ConstantTok}[1]{\textcolor[rgb]{0.53,0.00,0.00}{{#1}}}
    \newcommand{\SpecialCharTok}[1]{\textcolor[rgb]{0.25,0.44,0.63}{{#1}}}
    \newcommand{\VerbatimStringTok}[1]{\textcolor[rgb]{0.25,0.44,0.63}{{#1}}}
    \newcommand{\SpecialStringTok}[1]{\textcolor[rgb]{0.73,0.40,0.53}{{#1}}}
    \newcommand{\ImportTok}[1]{{#1}}
    \newcommand{\DocumentationTok}[1]{\textcolor[rgb]{0.73,0.13,0.13}{\textit{{#1}}}}
    \newcommand{\AnnotationTok}[1]{\textcolor[rgb]{0.38,0.63,0.69}{\textbf{\textit{{#1}}}}}
    \newcommand{\CommentVarTok}[1]{\textcolor[rgb]{0.38,0.63,0.69}{\textbf{\textit{{#1}}}}}
    \newcommand{\VariableTok}[1]{\textcolor[rgb]{0.10,0.09,0.49}{{#1}}}
    \newcommand{\ControlFlowTok}[1]{\textcolor[rgb]{0.00,0.44,0.13}{\textbf{{#1}}}}
    \newcommand{\OperatorTok}[1]{\textcolor[rgb]{0.40,0.40,0.40}{{#1}}}
    \newcommand{\BuiltInTok}[1]{{#1}}
    \newcommand{\ExtensionTok}[1]{{#1}}
    \newcommand{\PreprocessorTok}[1]{\textcolor[rgb]{0.74,0.48,0.00}{{#1}}}
    \newcommand{\AttributeTok}[1]{\textcolor[rgb]{0.49,0.56,0.16}{{#1}}}
    \newcommand{\InformationTok}[1]{\textcolor[rgb]{0.38,0.63,0.69}{\textbf{\textit{{#1}}}}}
    \newcommand{\WarningTok}[1]{\textcolor[rgb]{0.38,0.63,0.69}{\textbf{\textit{{#1}}}}}
    
    
    % Define a nice break command that doesn't care if a line doesn't already
    % exist.
    \def\br{\hspace*{\fill} \\* }
    % Math Jax compatability definitions
    \def\gt{>}
    \def\lt{<}
    % Document parameters
    \title{Lab1}
    
    
    

    % Pygments definitions
    
\makeatletter
\def\PY@reset{\let\PY@it=\relax \let\PY@bf=\relax%
    \let\PY@ul=\relax \let\PY@tc=\relax%
    \let\PY@bc=\relax \let\PY@ff=\relax}
\def\PY@tok#1{\csname PY@tok@#1\endcsname}
\def\PY@toks#1+{\ifx\relax#1\empty\else%
    \PY@tok{#1}\expandafter\PY@toks\fi}
\def\PY@do#1{\PY@bc{\PY@tc{\PY@ul{%
    \PY@it{\PY@bf{\PY@ff{#1}}}}}}}
\def\PY#1#2{\PY@reset\PY@toks#1+\relax+\PY@do{#2}}

\expandafter\def\csname PY@tok@w\endcsname{\def\PY@tc##1{\textcolor[rgb]{0.73,0.73,0.73}{##1}}}
\expandafter\def\csname PY@tok@c\endcsname{\let\PY@it=\textit\def\PY@tc##1{\textcolor[rgb]{0.25,0.50,0.50}{##1}}}
\expandafter\def\csname PY@tok@cp\endcsname{\def\PY@tc##1{\textcolor[rgb]{0.74,0.48,0.00}{##1}}}
\expandafter\def\csname PY@tok@k\endcsname{\let\PY@bf=\textbf\def\PY@tc##1{\textcolor[rgb]{0.00,0.50,0.00}{##1}}}
\expandafter\def\csname PY@tok@kp\endcsname{\def\PY@tc##1{\textcolor[rgb]{0.00,0.50,0.00}{##1}}}
\expandafter\def\csname PY@tok@kt\endcsname{\def\PY@tc##1{\textcolor[rgb]{0.69,0.00,0.25}{##1}}}
\expandafter\def\csname PY@tok@o\endcsname{\def\PY@tc##1{\textcolor[rgb]{0.40,0.40,0.40}{##1}}}
\expandafter\def\csname PY@tok@ow\endcsname{\let\PY@bf=\textbf\def\PY@tc##1{\textcolor[rgb]{0.67,0.13,1.00}{##1}}}
\expandafter\def\csname PY@tok@nb\endcsname{\def\PY@tc##1{\textcolor[rgb]{0.00,0.50,0.00}{##1}}}
\expandafter\def\csname PY@tok@nf\endcsname{\def\PY@tc##1{\textcolor[rgb]{0.00,0.00,1.00}{##1}}}
\expandafter\def\csname PY@tok@nc\endcsname{\let\PY@bf=\textbf\def\PY@tc##1{\textcolor[rgb]{0.00,0.00,1.00}{##1}}}
\expandafter\def\csname PY@tok@nn\endcsname{\let\PY@bf=\textbf\def\PY@tc##1{\textcolor[rgb]{0.00,0.00,1.00}{##1}}}
\expandafter\def\csname PY@tok@ne\endcsname{\let\PY@bf=\textbf\def\PY@tc##1{\textcolor[rgb]{0.82,0.25,0.23}{##1}}}
\expandafter\def\csname PY@tok@nv\endcsname{\def\PY@tc##1{\textcolor[rgb]{0.10,0.09,0.49}{##1}}}
\expandafter\def\csname PY@tok@no\endcsname{\def\PY@tc##1{\textcolor[rgb]{0.53,0.00,0.00}{##1}}}
\expandafter\def\csname PY@tok@nl\endcsname{\def\PY@tc##1{\textcolor[rgb]{0.63,0.63,0.00}{##1}}}
\expandafter\def\csname PY@tok@ni\endcsname{\let\PY@bf=\textbf\def\PY@tc##1{\textcolor[rgb]{0.60,0.60,0.60}{##1}}}
\expandafter\def\csname PY@tok@na\endcsname{\def\PY@tc##1{\textcolor[rgb]{0.49,0.56,0.16}{##1}}}
\expandafter\def\csname PY@tok@nt\endcsname{\let\PY@bf=\textbf\def\PY@tc##1{\textcolor[rgb]{0.00,0.50,0.00}{##1}}}
\expandafter\def\csname PY@tok@nd\endcsname{\def\PY@tc##1{\textcolor[rgb]{0.67,0.13,1.00}{##1}}}
\expandafter\def\csname PY@tok@s\endcsname{\def\PY@tc##1{\textcolor[rgb]{0.73,0.13,0.13}{##1}}}
\expandafter\def\csname PY@tok@sd\endcsname{\let\PY@it=\textit\def\PY@tc##1{\textcolor[rgb]{0.73,0.13,0.13}{##1}}}
\expandafter\def\csname PY@tok@si\endcsname{\let\PY@bf=\textbf\def\PY@tc##1{\textcolor[rgb]{0.73,0.40,0.53}{##1}}}
\expandafter\def\csname PY@tok@se\endcsname{\let\PY@bf=\textbf\def\PY@tc##1{\textcolor[rgb]{0.73,0.40,0.13}{##1}}}
\expandafter\def\csname PY@tok@sr\endcsname{\def\PY@tc##1{\textcolor[rgb]{0.73,0.40,0.53}{##1}}}
\expandafter\def\csname PY@tok@ss\endcsname{\def\PY@tc##1{\textcolor[rgb]{0.10,0.09,0.49}{##1}}}
\expandafter\def\csname PY@tok@sx\endcsname{\def\PY@tc##1{\textcolor[rgb]{0.00,0.50,0.00}{##1}}}
\expandafter\def\csname PY@tok@m\endcsname{\def\PY@tc##1{\textcolor[rgb]{0.40,0.40,0.40}{##1}}}
\expandafter\def\csname PY@tok@gh\endcsname{\let\PY@bf=\textbf\def\PY@tc##1{\textcolor[rgb]{0.00,0.00,0.50}{##1}}}
\expandafter\def\csname PY@tok@gu\endcsname{\let\PY@bf=\textbf\def\PY@tc##1{\textcolor[rgb]{0.50,0.00,0.50}{##1}}}
\expandafter\def\csname PY@tok@gd\endcsname{\def\PY@tc##1{\textcolor[rgb]{0.63,0.00,0.00}{##1}}}
\expandafter\def\csname PY@tok@gi\endcsname{\def\PY@tc##1{\textcolor[rgb]{0.00,0.63,0.00}{##1}}}
\expandafter\def\csname PY@tok@gr\endcsname{\def\PY@tc##1{\textcolor[rgb]{1.00,0.00,0.00}{##1}}}
\expandafter\def\csname PY@tok@ge\endcsname{\let\PY@it=\textit}
\expandafter\def\csname PY@tok@gs\endcsname{\let\PY@bf=\textbf}
\expandafter\def\csname PY@tok@gp\endcsname{\let\PY@bf=\textbf\def\PY@tc##1{\textcolor[rgb]{0.00,0.00,0.50}{##1}}}
\expandafter\def\csname PY@tok@go\endcsname{\def\PY@tc##1{\textcolor[rgb]{0.53,0.53,0.53}{##1}}}
\expandafter\def\csname PY@tok@gt\endcsname{\def\PY@tc##1{\textcolor[rgb]{0.00,0.27,0.87}{##1}}}
\expandafter\def\csname PY@tok@err\endcsname{\def\PY@bc##1{\setlength{\fboxsep}{0pt}\fcolorbox[rgb]{1.00,0.00,0.00}{1,1,1}{\strut ##1}}}
\expandafter\def\csname PY@tok@kc\endcsname{\let\PY@bf=\textbf\def\PY@tc##1{\textcolor[rgb]{0.00,0.50,0.00}{##1}}}
\expandafter\def\csname PY@tok@kd\endcsname{\let\PY@bf=\textbf\def\PY@tc##1{\textcolor[rgb]{0.00,0.50,0.00}{##1}}}
\expandafter\def\csname PY@tok@kn\endcsname{\let\PY@bf=\textbf\def\PY@tc##1{\textcolor[rgb]{0.00,0.50,0.00}{##1}}}
\expandafter\def\csname PY@tok@kr\endcsname{\let\PY@bf=\textbf\def\PY@tc##1{\textcolor[rgb]{0.00,0.50,0.00}{##1}}}
\expandafter\def\csname PY@tok@bp\endcsname{\def\PY@tc##1{\textcolor[rgb]{0.00,0.50,0.00}{##1}}}
\expandafter\def\csname PY@tok@fm\endcsname{\def\PY@tc##1{\textcolor[rgb]{0.00,0.00,1.00}{##1}}}
\expandafter\def\csname PY@tok@vc\endcsname{\def\PY@tc##1{\textcolor[rgb]{0.10,0.09,0.49}{##1}}}
\expandafter\def\csname PY@tok@vg\endcsname{\def\PY@tc##1{\textcolor[rgb]{0.10,0.09,0.49}{##1}}}
\expandafter\def\csname PY@tok@vi\endcsname{\def\PY@tc##1{\textcolor[rgb]{0.10,0.09,0.49}{##1}}}
\expandafter\def\csname PY@tok@vm\endcsname{\def\PY@tc##1{\textcolor[rgb]{0.10,0.09,0.49}{##1}}}
\expandafter\def\csname PY@tok@sa\endcsname{\def\PY@tc##1{\textcolor[rgb]{0.73,0.13,0.13}{##1}}}
\expandafter\def\csname PY@tok@sb\endcsname{\def\PY@tc##1{\textcolor[rgb]{0.73,0.13,0.13}{##1}}}
\expandafter\def\csname PY@tok@sc\endcsname{\def\PY@tc##1{\textcolor[rgb]{0.73,0.13,0.13}{##1}}}
\expandafter\def\csname PY@tok@dl\endcsname{\def\PY@tc##1{\textcolor[rgb]{0.73,0.13,0.13}{##1}}}
\expandafter\def\csname PY@tok@s2\endcsname{\def\PY@tc##1{\textcolor[rgb]{0.73,0.13,0.13}{##1}}}
\expandafter\def\csname PY@tok@sh\endcsname{\def\PY@tc##1{\textcolor[rgb]{0.73,0.13,0.13}{##1}}}
\expandafter\def\csname PY@tok@s1\endcsname{\def\PY@tc##1{\textcolor[rgb]{0.73,0.13,0.13}{##1}}}
\expandafter\def\csname PY@tok@mb\endcsname{\def\PY@tc##1{\textcolor[rgb]{0.40,0.40,0.40}{##1}}}
\expandafter\def\csname PY@tok@mf\endcsname{\def\PY@tc##1{\textcolor[rgb]{0.40,0.40,0.40}{##1}}}
\expandafter\def\csname PY@tok@mh\endcsname{\def\PY@tc##1{\textcolor[rgb]{0.40,0.40,0.40}{##1}}}
\expandafter\def\csname PY@tok@mi\endcsname{\def\PY@tc##1{\textcolor[rgb]{0.40,0.40,0.40}{##1}}}
\expandafter\def\csname PY@tok@il\endcsname{\def\PY@tc##1{\textcolor[rgb]{0.40,0.40,0.40}{##1}}}
\expandafter\def\csname PY@tok@mo\endcsname{\def\PY@tc##1{\textcolor[rgb]{0.40,0.40,0.40}{##1}}}
\expandafter\def\csname PY@tok@ch\endcsname{\let\PY@it=\textit\def\PY@tc##1{\textcolor[rgb]{0.25,0.50,0.50}{##1}}}
\expandafter\def\csname PY@tok@cm\endcsname{\let\PY@it=\textit\def\PY@tc##1{\textcolor[rgb]{0.25,0.50,0.50}{##1}}}
\expandafter\def\csname PY@tok@cpf\endcsname{\let\PY@it=\textit\def\PY@tc##1{\textcolor[rgb]{0.25,0.50,0.50}{##1}}}
\expandafter\def\csname PY@tok@c1\endcsname{\let\PY@it=\textit\def\PY@tc##1{\textcolor[rgb]{0.25,0.50,0.50}{##1}}}
\expandafter\def\csname PY@tok@cs\endcsname{\let\PY@it=\textit\def\PY@tc##1{\textcolor[rgb]{0.25,0.50,0.50}{##1}}}

\def\PYZbs{\char`\\}
\def\PYZus{\char`\_}
\def\PYZob{\char`\{}
\def\PYZcb{\char`\}}
\def\PYZca{\char`\^}
\def\PYZam{\char`\&}
\def\PYZlt{\char`\<}
\def\PYZgt{\char`\>}
\def\PYZsh{\char`\#}
\def\PYZpc{\char`\%}
\def\PYZdl{\char`\$}
\def\PYZhy{\char`\-}
\def\PYZsq{\char`\'}
\def\PYZdq{\char`\"}
\def\PYZti{\char`\~}
% for compatibility with earlier versions
\def\PYZat{@}
\def\PYZlb{[}
\def\PYZrb{]}
\makeatother


    % Exact colors from NB
    \definecolor{incolor}{rgb}{0.0, 0.0, 0.5}
    \definecolor{outcolor}{rgb}{0.545, 0.0, 0.0}



    
    % Prevent overflowing lines due to hard-to-break entities
    \sloppy 
    % Setup hyperref package
    \hypersetup{
      breaklinks=true,  % so long urls are correctly broken across lines
      colorlinks=true,
      urlcolor=urlcolor,
      linkcolor=linkcolor,
      citecolor=citecolor,
      }
    % Slightly bigger margins than the latex defaults
    
    \geometry{verbose,tmargin=1in,bmargin=1in,lmargin=1in,rmargin=1in}
    
    

    \begin{document}
    
    
    \maketitle
    
    

    
    \section{Heart Disease - Lab 1}\label{heart-disease---lab-1}

Martin Garcia, Joanna Duran, Daniel Byrne

    \begin{Verbatim}[commandchars=\\\{\}]
{\color{incolor}In [{\color{incolor}402}]:} \PY{c+c1}{\PYZsh{}packages}
          \PY{k+kn}{import} \PY{n+nn}{pandas} \PY{k}{as} \PY{n+nn}{pd}
          \PY{k+kn}{import} \PY{n+nn}{numpy} \PY{k}{as} \PY{n+nn}{np}
          \PY{k+kn}{from} \PY{n+nn}{IPython}\PY{n+nn}{.}\PY{n+nn}{display} \PY{k}{import} \PY{n}{display}
          \PY{k+kn}{from} \PY{n+nn}{IPython}\PY{n+nn}{.}\PY{n+nn}{display} \PY{k}{import} \PY{n}{HTML}
          
          \PY{c+c1}{\PYZsh{} plotting}
          \PY{k+kn}{import} \PY{n+nn}{matplotlib}\PY{n+nn}{.}\PY{n+nn}{pyplot} \PY{k}{as} \PY{n+nn}{plt}
          \PY{k+kn}{import} \PY{n+nn}{seaborn} \PY{k}{as} \PY{n+nn}{sns}
          \PY{n}{sns}\PY{o}{.}\PY{n}{set}\PY{p}{(}\PY{n}{color\PYZus{}codes}\PY{o}{=}\PY{k+kc}{True}\PY{p}{)}
          
          
          \PY{k+kn}{from} \PY{n+nn}{sklearn}\PY{n+nn}{.}\PY{n+nn}{model\PYZus{}selection} \PY{k}{import} \PY{n}{train\PYZus{}test\PYZus{}split}
          \PY{k+kn}{from} \PY{n+nn}{sklearn}\PY{n+nn}{.}\PY{n+nn}{linear\PYZus{}model} \PY{k}{import} \PY{n}{LogisticRegression}
          \PY{k+kn}{from} \PY{n+nn}{sklearn}\PY{n+nn}{.}\PY{n+nn}{metrics} \PY{k}{import} \PY{n}{classification\PYZus{}report}
          \PY{k+kn}{from} \PY{n+nn}{sklearn}\PY{n+nn}{.}\PY{n+nn}{metrics} \PY{k}{import} \PY{n}{confusion\PYZus{}matrix}
          \PY{k+kn}{import} \PY{n+nn}{statsmodels}\PY{n+nn}{.}\PY{n+nn}{api} \PY{k}{as} \PY{n+nn}{sm}
          
          \PY{k+kn}{from} \PY{n+nn}{IPython}\PY{n+nn}{.}\PY{n+nn}{core}\PY{n+nn}{.}\PY{n+nn}{interactiveshell} \PY{k}{import} \PY{n}{InteractiveShell}
          \PY{n}{InteractiveShell}\PY{o}{.}\PY{n}{ast\PYZus{}node\PYZus{}interactivity} \PY{o}{=} \PY{l+s+s2}{\PYZdq{}}\PY{l+s+s2}{all}\PY{l+s+s2}{\PYZdq{}}
          
          \PY{k+kn}{from} \PY{n+nn}{scipy} \PY{k}{import} \PY{n}{stats}
          \PY{k+kn}{import} \PY{n+nn}{warnings}
          \PY{n}{warnings}\PY{o}{.}\PY{n}{filterwarnings}\PY{p}{(}\PY{l+s+s1}{\PYZsq{}}\PY{l+s+s1}{ignore}\PY{l+s+s1}{\PYZsq{}}\PY{p}{)}
          
          
          
          \PY{c+c1}{\PYZsh{}read the csv file}
          \PY{n}{df} \PY{o}{=} \PY{n}{pd}\PY{o}{.}\PY{n}{read\PYZus{}csv}\PY{p}{(}\PY{l+s+s1}{\PYZsq{}}\PY{l+s+s1}{heart.csv}\PY{l+s+s1}{\PYZsq{}}\PY{p}{)}
\end{Verbatim}


    \subsubsection{Business Understanding}\label{business-understanding}

    The Cleveland Heart Disease Database, CHDD, produced in partnership with
the V.A. Medical Center in Long Beach and Cleveland Clinic Foundation
has grown to be a popular repository of heart disease specific
measurable attributes. The
\href{https://archive.ics.uci.edu/ml/datasets/Heart+Disease}{machine
learning repository} hosting the dataset boasts 58 paper's citing it.

The CHDD database has been used as the basis of multiple models of
various complexity since its creation in 1998. The original model using
logistic regression boasted a 77\% accuracy in predicting heart disease.
Subsequent models have pushed that accuracy metric up past 80\% and to
as high as 89\% in one model.

    \subsubsection{Data Meaning Type}\label{data-meaning-type}

The original CHDD dataset contains 76 attributes, but most published
experiments, including the original, rely only 13 attributes from the
dataset. We use data mining techniques to attempt to build a model to
that tries to predict coronary artery disease, CAD, in patients from the
13 commonly used attribute from the dataset. A description of the data
points used in this model is in Table 1.

The data used to derive the model originated from data collected from
303 consecutive patients slated to receive a coronary angiography at the
Cleveland Clinic between May 1981 and September 1984. None of the
patients had a prior history or evidence of prior heart attack or known
vascular or cardiomyopathic disease. Each patient provided a medical
history and underwent a physical examination, provided electrocardiogram
at rest, and provided a serum cholesterol fasting blood sugar
measurement as part of a routine evaluation.

Historical data were recorded and coded without knowledge of the
additional test data. The patients also underwent 3 noninvasive tests,
and exercise electrocardiogram, a thallium scintigraphy and a cardiac
fluoroscopy. The results of these tests were not interpreted until after
the angiogram was read.

    \subparagraph{Table 1 - Data Parameter
Descriptions}\label{table-1---data-parameter-descriptions}

\begin{longtable}[]{@{}lll@{}}
\toprule
\begin{minipage}[b]{0.09\columnwidth}\raggedright\strut
Data\strut
\end{minipage} & \begin{minipage}[b]{0.44\columnwidth}\raggedright\strut
Description\strut
\end{minipage} & \begin{minipage}[b]{0.38\columnwidth}\raggedright\strut
Attributes / Attribute Type / Scale / Values\strut
\end{minipage}\tabularnewline
\midrule
\endhead
\begin{minipage}[t]{0.09\columnwidth}\raggedright\strut
Age\strut
\end{minipage} & \begin{minipage}[t]{0.44\columnwidth}\raggedright\strut
Age in years\strut
\end{minipage} & \begin{minipage}[t]{0.38\columnwidth}\raggedright\strut
range 29 - 77 / numeric - ratio / continuous / int64\strut
\end{minipage}\tabularnewline
\begin{minipage}[t]{0.09\columnwidth}\raggedright\strut
sex\strut
\end{minipage} & \begin{minipage}[t]{0.44\columnwidth}\raggedright\strut
Gender\strut
\end{minipage} & \begin{minipage}[t]{0.38\columnwidth}\raggedright\strut
Male: 1 Female: 0 / categorical - nominal / discrete + binary /
int64\strut
\end{minipage}\tabularnewline
\begin{minipage}[t]{0.09\columnwidth}\raggedright\strut
cp\strut
\end{minipage} & \begin{minipage}[t]{0.44\columnwidth}\raggedright\strut
Angina Categories / chest pain type\strut
\end{minipage} & \begin{minipage}[t]{0.38\columnwidth}\raggedright\strut
Typical:0, Atypical:1, Non-Anginal:2, Asymptomatic:3 /
categorical-ordinal / discrete\strut
\end{minipage}\tabularnewline
\begin{minipage}[t]{0.09\columnwidth}\raggedright\strut
trestbps\strut
\end{minipage} & \begin{minipage}[t]{0.44\columnwidth}\raggedright\strut
Resting Blood Pressure (mm Hg - unit of pressure )\strut
\end{minipage} & \begin{minipage}[t]{0.38\columnwidth}\raggedright\strut
range 94- 200 / numeric - / continuous / int64\strut
\end{minipage}\tabularnewline
\begin{minipage}[t]{0.09\columnwidth}\raggedright\strut
chol\strut
\end{minipage} & \begin{minipage}[t]{0.44\columnwidth}\raggedright\strut
Serum Cholesterol and Triglycerides (mg/dl - milligrams per
deciliter)\strut
\end{minipage} & \begin{minipage}[t]{0.38\columnwidth}\raggedright\strut
range 126 - 564 / numeric / continuous / int64\strut
\end{minipage}\tabularnewline
\begin{minipage}[t]{0.09\columnwidth}\raggedright\strut
fbs\strut
\end{minipage} & \begin{minipage}[t]{0.44\columnwidth}\raggedright\strut
Fasting Blood Sugar \textgreater{} 120 (mg/dl)\strut
\end{minipage} & \begin{minipage}[t]{0.38\columnwidth}\raggedright\strut
True:1 False:0 / discrete + binary / int64\strut
\end{minipage}\tabularnewline
\begin{minipage}[t]{0.09\columnwidth}\raggedright\strut
restecg\strut
\end{minipage} & \begin{minipage}[t]{0.44\columnwidth}\raggedright\strut
Resting Electrocardiogram\strut
\end{minipage} & \begin{minipage}[t]{0.38\columnwidth}\raggedright\strut
Normal: 0, Abnormal:1, 2:Probable Ventricular Hypertrophy / thickening
heart walls\strut
\end{minipage}\tabularnewline
\begin{minipage}[t]{0.09\columnwidth}\raggedright\strut
thalach\strut
\end{minipage} & \begin{minipage}[t]{0.44\columnwidth}\raggedright\strut
Maximum Heart Rate\strut
\end{minipage} & \begin{minipage}[t]{0.38\columnwidth}\raggedright\strut
range 71- 202 / numeric / continuous / int64\strut
\end{minipage}\tabularnewline
\begin{minipage}[t]{0.09\columnwidth}\raggedright\strut
exang\strut
\end{minipage} & \begin{minipage}[t]{0.44\columnwidth}\raggedright\strut
Exercise Induced Angina\strut
\end{minipage} & \begin{minipage}[t]{0.38\columnwidth}\raggedright\strut
Yes: 1, No: 0 / categorical- / discrete + binary\strut
\end{minipage}\tabularnewline
\begin{minipage}[t]{0.09\columnwidth}\raggedright\strut
oldpeak\strut
\end{minipage} & \begin{minipage}[t]{0.44\columnwidth}\raggedright\strut
ST Depression Induced by Exercise Relative to Rest\strut
\end{minipage} & \begin{minipage}[t]{0.38\columnwidth}\raggedright\strut
range 0 - 6.2 / numeric / float64\strut
\end{minipage}\tabularnewline
\begin{minipage}[t]{0.09\columnwidth}\raggedright\strut
slope\strut
\end{minipage} & \begin{minipage}[t]{0.44\columnwidth}\raggedright\strut
Slope of the Peak Exercise ST Segment\strut
\end{minipage} & \begin{minipage}[t]{0.38\columnwidth}\raggedright\strut
Upsloping:1, Flat:2, Downsloping:3 / int64\strut
\end{minipage}\tabularnewline
\begin{minipage}[t]{0.09\columnwidth}\raggedright\strut
ca\strut
\end{minipage} & \begin{minipage}[t]{0.44\columnwidth}\raggedright\strut
Number of Major Vessels Colored by Fluoroscopy\strut
\end{minipage} & \begin{minipage}[t]{0.38\columnwidth}\raggedright\strut
Count with range 0-3 / int64\strut
\end{minipage}\tabularnewline
\begin{minipage}[t]{0.09\columnwidth}\raggedright\strut
thal\strut
\end{minipage} & \begin{minipage}[t]{0.44\columnwidth}\raggedright\strut
Thalassemia - blood disorder, less hemoglobin and red blood counts\strut
\end{minipage} & \begin{minipage}[t]{0.38\columnwidth}\raggedright\strut
3: Normal, 6: fixed defect, 7:reversable defect\strut
\end{minipage}\tabularnewline
\begin{minipage}[t]{0.09\columnwidth}\raggedright\strut
target\strut
\end{minipage} & \begin{minipage}[t]{0.44\columnwidth}\raggedright\strut
Heart Disease Present in Patient\strut
\end{minipage} & \begin{minipage}[t]{0.38\columnwidth}\raggedright\strut
Yes: 1, No: 0 / discrete+ binary / int64\strut
\end{minipage}\tabularnewline
\bottomrule
\end{longtable}

    \subsubsection{Data Quality}\label{data-quality}

    \paragraph{MIssing Values}\label{missing-values}

In the 13 variables used in this evaluation, the dataset boasts no
missing variables.

    \begin{Verbatim}[commandchars=\\\{\}]
{\color{incolor}In [{\color{incolor}367}]:} \PY{c+c1}{\PYZsh{}heatmap for missing values}
          \PY{n}{sns}\PY{o}{.}\PY{n}{heatmap}\PY{p}{(}\PY{n}{df}\PY{o}{.}\PY{n}{isnull}\PY{p}{(}\PY{p}{)}\PY{p}{)}\PY{p}{;}
          \PY{n}{plt}\PY{o}{.}\PY{n}{xlabel}\PY{p}{(}\PY{l+s+s2}{\PYZdq{}}\PY{l+s+s2}{Dataset Variables}\PY{l+s+s2}{\PYZdq{}}\PY{p}{)}\PY{p}{;}
\end{Verbatim}


    \begin{center}
    \adjustimage{max size={0.9\linewidth}{0.9\paperheight}}{output_8_0.png}
    \end{center}
    { \hspace*{\fill} \\}
    
    \paragraph{Data Descriptions}\label{data-descriptions}

Upon initial overview the dataset seemed very robust since every row has
an entry for every attribute. While examining the data there seems to be
some discrepancies in the descriptions provided from Kaggle as well as
the original data source.

\begin{itemize}
\tightlist
\item
  For the category "slope", the provided description "Upsloping:1,
  Flat:2, Downsloping:3" does not match the data set values of "0,1,2".
\item
  This represeents the same range shifted to 0
\item
  For the category "ca", the Kagle description "Count with range 0-3"
  does not match the data set values of "0-4".
\item
  For the category "thal", the provided description "3: Normal, 6: fixed
  defect, 7: reversable defect" does not match the data set values of
  "0,1,2,3".
\end{itemize}

We investigated the Kaggle discussion boards to see if anybody else had
answers. Some of the questions were also asked but no resolution was
given.

    \paragraph{Outlier analysis}\label{outlier-analysis}

We plotted the continuous data points to get an intial sense of the
potential outliers in the data.

    \begin{Verbatim}[commandchars=\\\{\}]
{\color{incolor}In [{\color{incolor}368}]:} \PY{c+c1}{\PYZsh{} matplotlib histograms of continuous variables}
          \PY{n}{plt}\PY{o}{.}\PY{n}{figure}\PY{p}{(}\PY{n}{figsize}\PY{o}{=}\PY{p}{(}\PY{l+m+mi}{20}\PY{p}{,}\PY{l+m+mi}{10}\PY{p}{)}\PY{p}{)}
          \PY{n}{plt}\PY{o}{.}\PY{n}{subplot}\PY{p}{(}\PY{l+m+mi}{2}\PY{p}{,}\PY{l+m+mi}{2}\PY{p}{,}\PY{l+m+mi}{1}\PY{p}{)}
          \PY{n}{sns}\PY{o}{.}\PY{n}{boxplot}\PY{p}{(}\PY{n}{x}\PY{o}{=}\PY{n}{df}\PY{p}{[}\PY{l+s+s2}{\PYZdq{}}\PY{l+s+s2}{age}\PY{l+s+s2}{\PYZdq{}}\PY{p}{]}\PY{p}{)}
          \PY{n}{plt}\PY{o}{.}\PY{n}{title}\PY{p}{(}\PY{l+s+s1}{\PYZsq{}}\PY{l+s+s1}{Age}\PY{l+s+s1}{\PYZsq{}}\PY{p}{)}
          
          \PY{n}{plt}\PY{o}{.}\PY{n}{subplot}\PY{p}{(}\PY{l+m+mi}{2}\PY{p}{,}\PY{l+m+mi}{2}\PY{p}{,}\PY{l+m+mi}{2}\PY{p}{)}
          \PY{n}{sns}\PY{o}{.}\PY{n}{boxplot}\PY{p}{(}\PY{n}{x}\PY{o}{=}\PY{n}{df}\PY{p}{[}\PY{l+s+s2}{\PYZdq{}}\PY{l+s+s2}{trestbps}\PY{l+s+s2}{\PYZdq{}}\PY{p}{]}\PY{p}{)}
          \PY{n}{plt}\PY{o}{.}\PY{n}{title}\PY{p}{(}\PY{l+s+s1}{\PYZsq{}}\PY{l+s+s1}{Resting BPS}\PY{l+s+s1}{\PYZsq{}}\PY{p}{)}
          
          \PY{n}{plt}\PY{o}{.}\PY{n}{subplot}\PY{p}{(}\PY{l+m+mi}{2}\PY{p}{,}\PY{l+m+mi}{2}\PY{p}{,}\PY{l+m+mi}{3}\PY{p}{)}
          \PY{n}{sns}\PY{o}{.}\PY{n}{boxplot}\PY{p}{(}\PY{n}{x}\PY{o}{=}\PY{n}{df}\PY{p}{[}\PY{l+s+s2}{\PYZdq{}}\PY{l+s+s2}{chol}\PY{l+s+s2}{\PYZdq{}}\PY{p}{]}\PY{p}{)}
          \PY{n}{plt}\PY{o}{.}\PY{n}{title}\PY{p}{(}\PY{l+s+s1}{\PYZsq{}}\PY{l+s+s1}{Cholesterol}\PY{l+s+s1}{\PYZsq{}}\PY{p}{)}
          
          \PY{n}{plt}\PY{o}{.}\PY{n}{subplot}\PY{p}{(}\PY{l+m+mi}{2}\PY{p}{,}\PY{l+m+mi}{2}\PY{p}{,}\PY{l+m+mi}{4}\PY{p}{)}
          \PY{n}{sns}\PY{o}{.}\PY{n}{boxplot}\PY{p}{(}\PY{n}{x}\PY{o}{=}\PY{n}{df}\PY{p}{[}\PY{l+s+s2}{\PYZdq{}}\PY{l+s+s2}{thalach}\PY{l+s+s2}{\PYZdq{}}\PY{p}{]}\PY{p}{)}
          \PY{n}{plt}\PY{o}{.}\PY{n}{title}\PY{p}{(}\PY{l+s+s1}{\PYZsq{}}\PY{l+s+s1}{Max HR}\PY{l+s+s1}{\PYZsq{}}\PY{p}{)}
          \PY{n}{plt}\PY{o}{.}\PY{n}{show}\PY{p}{(}\PY{p}{)}\PY{p}{;}
              
\end{Verbatim}


    \begin{center}
    \adjustimage{max size={0.9\linewidth}{0.9\paperheight}}{output_11_0.png}
    \end{center}
    { \hspace*{\fill} \\}
    
    The Serum Cholesterol and Triglycerides, chol, and Resting BPS, trestbps
parameters have slightly left skewed distributions influenced by a small
set of outliers on the high end of each variables range.

The top 4 values of chol are all in the 99\%. Thus these 4 values are
substantially larger than the 2Q value and thus in this dataset are
outliers. These values will pull the mean up, and thus statistical
analysis methods tools that use the mean to set confidence intervals and
make predictions will be affected by these data points.

    \begin{Verbatim}[commandchars=\\\{\}]
{\color{incolor}In [{\color{incolor}369}]:} \PY{k}{def} \PY{n+nf}{mypercentile}\PY{p}{(}\PY{n}{score}\PY{p}{)}\PY{p}{:}
              \PY{k}{return} \PY{n}{stats}\PY{o}{.}\PY{n}{percentileofscore}\PY{p}{(}\PY{n}{df}\PY{p}{,}\PY{n}{score}\PY{p}{)}
          
          
          \PY{c+c1}{\PYZsh{} top 5 chol levels  }
          \PY{n}{df}\PY{o}{.}\PY{n}{sort\PYZus{}values}\PY{p}{(}\PY{n}{by}\PY{o}{=}\PY{p}{[}\PY{l+s+s1}{\PYZsq{}}\PY{l+s+s1}{chol}\PY{l+s+s1}{\PYZsq{}}\PY{p}{]}\PY{p}{,}\PY{n}{ascending}\PY{o}{=}\PY{k+kc}{False}\PY{p}{)}\PY{o}{.}\PY{n}{head}\PY{p}{(}\PY{l+m+mi}{4}\PY{p}{)}\PY{p}{[}\PY{p}{[}\PY{l+s+s2}{\PYZdq{}}\PY{l+s+s2}{chol}\PY{l+s+s2}{\PYZdq{}}\PY{p}{]}\PY{p}{]}
          
          \PY{n}{nearest} \PY{o}{=} \PY{n}{np}\PY{o}{.}\PY{n}{percentile}\PY{p}{(}\PY{n}{df}\PY{p}{[}\PY{p}{[}\PY{l+s+s2}{\PYZdq{}}\PY{l+s+s2}{chol}\PY{l+s+s2}{\PYZdq{}}\PY{p}{]}\PY{p}{]}\PY{p}{,}\PY{l+m+mi}{99}\PY{p}{,}\PY{n}{interpolation}\PY{o}{=}\PY{l+s+s1}{\PYZsq{}}\PY{l+s+s1}{nearest}\PY{l+s+s1}{\PYZsq{}}\PY{p}{)}
          \PY{n+nb}{print}\PY{p}{(}\PY{l+s+s1}{\PYZsq{}}\PY{l+s+s1}{99}\PY{l+s+s1}{\PYZpc{}}\PY{l+s+s1}{ Nearest Neighbor Data Point :}\PY{l+s+s1}{\PYZsq{}}\PY{p}{,} \PY{n+nb}{str}\PY{p}{(}\PY{n}{nearest}\PY{p}{)}\PY{p}{)}
\end{Verbatim}


\begin{Verbatim}[commandchars=\\\{\}]
{\color{outcolor}Out[{\color{outcolor}369}]:}      chol
          85    564
          28    417
          246   409
          220   407
\end{Verbatim}
            
    \begin{Verbatim}[commandchars=\\\{\}]
99\% Nearest Neighbor Data Point : 407

    \end{Verbatim}

    Likewise The top 4 values of trestbps are all in the 99\% as well. Thus
these 4 values are substantially larger than the 2Q value and thus in
this dataset are outliers. These values will pull the mean up, and thus
statistical analysis methods tools that use the mean to set confidence
intervals and make predictions will be affected by these data points.

    \begin{Verbatim}[commandchars=\\\{\}]
{\color{incolor}In [{\color{incolor}370}]:} \PY{c+c1}{\PYZsh{} top 5 trestbps levels}
          \PY{n}{df}\PY{o}{.}\PY{n}{sort\PYZus{}values}\PY{p}{(}\PY{n}{by}\PY{o}{=}\PY{p}{[}\PY{l+s+s1}{\PYZsq{}}\PY{l+s+s1}{trestbps}\PY{l+s+s1}{\PYZsq{}}\PY{p}{]}\PY{p}{,}\PY{n}{ascending}\PY{o}{=}\PY{k+kc}{False}\PY{p}{)}\PY{o}{.}\PY{n}{head}\PY{p}{(}\PY{l+m+mi}{4}\PY{p}{)}\PY{p}{[}\PY{p}{[}\PY{l+s+s2}{\PYZdq{}}\PY{l+s+s2}{trestbps}\PY{l+s+s2}{\PYZdq{}}\PY{p}{]}\PY{p}{]}
          \PY{n}{nearest} \PY{o}{=} \PY{n}{np}\PY{o}{.}\PY{n}{percentile}\PY{p}{(}\PY{n}{df}\PY{p}{[}\PY{p}{[}\PY{l+s+s2}{\PYZdq{}}\PY{l+s+s2}{trestbps}\PY{l+s+s2}{\PYZdq{}}\PY{p}{]}\PY{p}{]}\PY{p}{,}\PY{l+m+mi}{99}\PY{p}{,}\PY{n}{interpolation}\PY{o}{=}\PY{l+s+s1}{\PYZsq{}}\PY{l+s+s1}{nearest}\PY{l+s+s1}{\PYZsq{}}\PY{p}{)}
          \PY{n+nb}{print}\PY{p}{(}\PY{l+s+s1}{\PYZsq{}}\PY{l+s+s1}{99}\PY{l+s+s1}{\PYZpc{}}\PY{l+s+s1}{ Nearest Neighbor Data Point :}\PY{l+s+s1}{\PYZsq{}}\PY{p}{,} \PY{n+nb}{str}\PY{p}{(}\PY{n}{nearest}\PY{p}{)}\PY{p}{)}
\end{Verbatim}


\begin{Verbatim}[commandchars=\\\{\}]
{\color{outcolor}Out[{\color{outcolor}370}]:}      trestbps
          223       200
          248       192
          266       180
          110       180
\end{Verbatim}
            
    \begin{Verbatim}[commandchars=\\\{\}]
99\% Nearest Neighbor Data Point : 180

    \end{Verbatim}

    \subsubsection{Simple Statistics}\label{simple-statistics}

    The summary statistics of the continuous variables show that the average
and the median age of each distribution are relatively close which is an
indication of a tight variance. They also all seem to be relatively
normally distributed. These characteristics opens them up for
traditional statistical analysis methodologies.

    \begin{Verbatim}[commandchars=\\\{\}]
{\color{incolor}In [{\color{incolor}371}]:} \PY{c+c1}{\PYZsh{}Summary statistics for all observations/patients}
          \PY{n}{stats} \PY{o}{=} \PY{n}{df}\PY{o}{.}\PY{n}{describe}\PY{p}{(}\PY{p}{)}
          \PY{n}{contstats} \PY{o}{=} \PY{n}{stats}\PY{p}{[}\PY{p}{[}\PY{l+s+s2}{\PYZdq{}}\PY{l+s+s2}{age}\PY{l+s+s2}{\PYZdq{}}\PY{p}{,}\PY{l+s+s2}{\PYZdq{}}\PY{l+s+s2}{trestbps}\PY{l+s+s2}{\PYZdq{}}\PY{p}{,}\PY{l+s+s2}{\PYZdq{}}\PY{l+s+s2}{chol}\PY{l+s+s2}{\PYZdq{}}\PY{p}{,}\PY{l+s+s2}{\PYZdq{}}\PY{l+s+s2}{thalach}\PY{l+s+s2}{\PYZdq{}}\PY{p}{]}\PY{p}{]}\PY{p}{;}
          \PY{n}{contstats}
\end{Verbatim}


\begin{Verbatim}[commandchars=\\\{\}]
{\color{outcolor}Out[{\color{outcolor}371}]:}               age    trestbps        chol     thalach
          count  303.000000  303.000000  303.000000  303.000000
          mean    54.366337  131.623762  246.264026  149.646865
          std      9.082101   17.538143   51.830751   22.905161
          min     29.000000   94.000000  126.000000   71.000000
          25\%     47.500000  120.000000  211.000000  133.500000
          50\%     55.000000  130.000000  240.000000  153.000000
          75\%     61.000000  140.000000  274.500000  166.000000
          max     77.000000  200.000000  564.000000  202.000000
\end{Verbatim}
            
    \begin{Verbatim}[commandchars=\\\{\}]
{\color{incolor}In [{\color{incolor}372}]:} \PY{c+c1}{\PYZsh{} matplotlib histograms of continuous variables}
          \PY{n}{plt}\PY{o}{.}\PY{n}{figure}\PY{p}{(}\PY{n}{figsize}\PY{o}{=}\PY{p}{(}\PY{l+m+mi}{20}\PY{p}{,}\PY{l+m+mi}{10}\PY{p}{)}\PY{p}{)}
          \PY{k}{for} \PY{n}{i} \PY{o+ow}{in} \PY{n+nb}{range}\PY{p}{(}\PY{l+m+mi}{0}\PY{p}{,} \PY{l+m+mi}{4}\PY{p}{)}\PY{p}{:}
              
              
              \PY{c+c1}{\PYZsh{} Set up the boxes}
              \PY{n}{ax} \PY{o}{=} \PY{n}{plt}\PY{o}{.}\PY{n}{subplot}\PY{p}{(}\PY{l+m+mi}{2}\PY{p}{,} \PY{l+m+mi}{2}\PY{p}{,} \PY{n}{i} \PY{o}{+} \PY{l+m+mi}{1}\PY{p}{)}
              
              \PY{c+c1}{\PYZsh{} Draw the histograms}
              \PY{k}{if} \PY{n}{i} \PY{o}{==} \PY{l+m+mi}{0}\PY{p}{:}
                  \PY{n}{ax}\PY{o}{.}\PY{n}{hist}\PY{p}{(}\PY{n}{df}\PY{p}{[}\PY{l+s+s1}{\PYZsq{}}\PY{l+s+s1}{age}\PY{l+s+s1}{\PYZsq{}}\PY{p}{]}\PY{p}{,} \PY{n}{bins} \PY{o}{=} \PY{n+nb}{int}\PY{p}{(}\PY{l+m+mi}{180}\PY{o}{/}\PY{l+m+mi}{15}\PY{p}{)}\PY{p}{,}\PY{n}{color} \PY{o}{=} \PY{l+s+s1}{\PYZsq{}}\PY{l+s+s1}{blue}\PY{l+s+s1}{\PYZsq{}}\PY{p}{,} \PY{n}{edgecolor} \PY{o}{=} \PY{l+s+s1}{\PYZsq{}}\PY{l+s+s1}{black}\PY{l+s+s1}{\PYZsq{}}\PY{p}{)}\PY{p}{;}
                  \PY{n}{ax}\PY{o}{.}\PY{n}{set\PYZus{}title}\PY{p}{(}\PY{l+s+s1}{\PYZsq{}}\PY{l+s+s1}{Age of Participants}\PY{l+s+s1}{\PYZsq{}}\PY{p}{,} \PY{n}{size} \PY{o}{=} \PY{l+m+mi}{12}\PY{p}{)}
                  \PY{n}{ax}\PY{o}{.}\PY{n}{set\PYZus{}xlabel}\PY{p}{(}\PY{l+s+s1}{\PYZsq{}}\PY{l+s+s1}{Age}\PY{l+s+s1}{\PYZsq{}}\PY{p}{)}
              
              \PY{k}{if} \PY{n}{i} \PY{o}{==} \PY{l+m+mi}{1}\PY{p}{:}
                  \PY{n}{ax}\PY{o}{.}\PY{n}{hist}\PY{p}{(}\PY{n}{df}\PY{p}{[}\PY{l+s+s1}{\PYZsq{}}\PY{l+s+s1}{trestbps}\PY{l+s+s1}{\PYZsq{}}\PY{p}{]}\PY{p}{,} \PY{n}{bins} \PY{o}{=} \PY{n+nb}{int}\PY{p}{(}\PY{l+m+mi}{180}\PY{o}{/}\PY{l+m+mi}{15}\PY{p}{)}\PY{p}{,}\PY{n}{color} \PY{o}{=} \PY{l+s+s1}{\PYZsq{}}\PY{l+s+s1}{green}\PY{l+s+s1}{\PYZsq{}}\PY{p}{,} \PY{n}{edgecolor} \PY{o}{=} \PY{l+s+s1}{\PYZsq{}}\PY{l+s+s1}{black}\PY{l+s+s1}{\PYZsq{}}\PY{p}{)}\PY{p}{;}
                  \PY{n}{ax}\PY{o}{.}\PY{n}{set\PYZus{}title}\PY{p}{(}\PY{l+s+s1}{\PYZsq{}}\PY{l+s+s1}{Resting Blood Pressure}\PY{l+s+s1}{\PYZsq{}}\PY{p}{,} \PY{n}{size} \PY{o}{=} \PY{l+m+mi}{12}\PY{p}{)}
                  \PY{n}{ax}\PY{o}{.}\PY{n}{set\PYZus{}xlabel}\PY{p}{(}\PY{l+s+s1}{\PYZsq{}}\PY{l+s+s1}{Resting BPS}\PY{l+s+s1}{\PYZsq{}}\PY{p}{)}
              
              \PY{k}{if} \PY{n}{i} \PY{o}{==} \PY{l+m+mi}{2}\PY{p}{:}
                  \PY{n}{ax}\PY{o}{.}\PY{n}{hist}\PY{p}{(}\PY{n}{df}\PY{p}{[}\PY{l+s+s1}{\PYZsq{}}\PY{l+s+s1}{chol}\PY{l+s+s1}{\PYZsq{}}\PY{p}{]}\PY{p}{,} \PY{n}{bins} \PY{o}{=} \PY{n+nb}{int}\PY{p}{(}\PY{l+m+mi}{180}\PY{o}{/}\PY{l+m+mi}{15}\PY{p}{)}\PY{p}{,}\PY{n}{color} \PY{o}{=} \PY{l+s+s1}{\PYZsq{}}\PY{l+s+s1}{yellow}\PY{l+s+s1}{\PYZsq{}}\PY{p}{,} \PY{n}{edgecolor} \PY{o}{=} \PY{l+s+s1}{\PYZsq{}}\PY{l+s+s1}{black}\PY{l+s+s1}{\PYZsq{}}\PY{p}{)}\PY{p}{;}
                  \PY{n}{ax}\PY{o}{.}\PY{n}{set\PYZus{}title}\PY{p}{(}\PY{l+s+s1}{\PYZsq{}}\PY{l+s+s1}{Cholesterol and Triglycerides}\PY{l+s+s1}{\PYZsq{}}\PY{p}{,} \PY{n}{size} \PY{o}{=} \PY{l+m+mi}{12}\PY{p}{)}
                  \PY{n}{ax}\PY{o}{.}\PY{n}{set\PYZus{}xlabel}\PY{p}{(}\PY{l+s+s1}{\PYZsq{}}\PY{l+s+s1}{Serum Cholesterol and Triglycerides}\PY{l+s+s1}{\PYZsq{}}\PY{p}{)}
                  
                  
              \PY{k}{if} \PY{n}{i} \PY{o}{==} \PY{l+m+mi}{3}\PY{p}{:}
                  \PY{n}{ax}\PY{o}{.}\PY{n}{hist}\PY{p}{(}\PY{n}{df}\PY{p}{[}\PY{l+s+s1}{\PYZsq{}}\PY{l+s+s1}{thalach}\PY{l+s+s1}{\PYZsq{}}\PY{p}{]}\PY{p}{,} \PY{n}{bins} \PY{o}{=} \PY{n+nb}{int}\PY{p}{(}\PY{l+m+mi}{180}\PY{o}{/}\PY{l+m+mi}{15}\PY{p}{)}\PY{p}{,}\PY{n}{color} \PY{o}{=} \PY{l+s+s1}{\PYZsq{}}\PY{l+s+s1}{red}\PY{l+s+s1}{\PYZsq{}}\PY{p}{,} \PY{n}{edgecolor} \PY{o}{=} \PY{l+s+s1}{\PYZsq{}}\PY{l+s+s1}{black}\PY{l+s+s1}{\PYZsq{}}\PY{p}{)}\PY{p}{;}
                  \PY{n}{ax}\PY{o}{.}\PY{n}{set\PYZus{}title}\PY{p}{(}\PY{l+s+s1}{\PYZsq{}}\PY{l+s+s1}{Maximum Heart Rate}\PY{l+s+s1}{\PYZsq{}}\PY{p}{,} \PY{n}{size} \PY{o}{=} \PY{l+m+mi}{12}\PY{p}{)}
                  \PY{n}{ax}\PY{o}{.}\PY{n}{set\PYZus{}xlabel}\PY{p}{(}\PY{l+s+s1}{\PYZsq{}}\PY{l+s+s1}{Heart Rate}\PY{l+s+s1}{\PYZsq{}}\PY{p}{)}
          
              \PY{c+c1}{\PYZsh{} Set the common y axis label}
              \PY{n}{ax}\PY{o}{.}\PY{n}{set\PYZus{}ylabel}\PY{p}{(}\PY{l+s+s1}{\PYZsq{}}\PY{l+s+s1}{Frequency}\PY{l+s+s1}{\PYZsq{}}\PY{p}{,} \PY{n}{size}\PY{o}{=} \PY{l+m+mi}{8}\PY{p}{)}
          
          \PY{n}{plt}\PY{o}{.}\PY{n}{tight\PYZus{}layout}\PY{p}{(}\PY{p}{)}\PY{p}{;}
          \PY{n}{plt}\PY{o}{.}\PY{n}{show}\PY{p}{(}\PY{p}{)}\PY{p}{;}
\end{Verbatim}


    \begin{center}
    \adjustimage{max size={0.9\linewidth}{0.9\paperheight}}{output_19_0.png}
    \end{center}
    { \hspace*{\fill} \\}
    
    \subsubsection{CAD Target Analysis}\label{cad-target-analysis}

There are more patients with the presence of heart disease (165) versus
no presence of heart disease (138), but this is still a well distributed
data set. However, the data does not represeent a random sample
considering study participants were chosen from a group of patients
reffered for coronary angiography. The assumption being that if a
petient is referred for this procedure there is at least a concern that
the patient referred could be at risk for CAD. Therefore, this study
cannot be generalized to a wider population.

    \begin{Verbatim}[commandchars=\\\{\}]
{\color{incolor}In [{\color{incolor}373}]:} \PY{c+c1}{\PYZsh{} Aggregate by CAD, presence:1 = yes 0 = no}
          \PY{n}{df\PYZus{}grouped} \PY{o}{=} \PY{n}{df}\PY{o}{.}\PY{n}{groupby}\PY{p}{(}\PY{n}{by}\PY{o}{=}\PY{l+s+s1}{\PYZsq{}}\PY{l+s+s1}{target}\PY{l+s+s1}{\PYZsq{}}\PY{p}{)}
          \PY{n+nb}{print} \PY{p}{(}\PY{n}{df\PYZus{}grouped}\PY{o}{.}\PY{n}{target}\PY{o}{.}\PY{n}{count}\PY{p}{(}\PY{p}{)}\PY{p}{)}
          \PY{n}{sns}\PY{o}{.}\PY{n}{countplot}\PY{p}{(}\PY{n}{x}\PY{o}{=}\PY{l+s+s1}{\PYZsq{}}\PY{l+s+s1}{target}\PY{l+s+s1}{\PYZsq{}}\PY{p}{,}\PY{n}{data}\PY{o}{=}\PY{n}{df}\PY{p}{)}\PY{p}{;}
\end{Verbatim}


    \begin{Verbatim}[commandchars=\\\{\}]
target
0    138
1    165
Name: target, dtype: int64

    \end{Verbatim}

    \begin{center}
    \adjustimage{max size={0.9\linewidth}{0.9\paperheight}}{output_21_1.png}
    \end{center}
    { \hspace*{\fill} \\}
    
    \textbf{Pearson Correlaion Coeficient Matrix}

    \begin{Verbatim}[commandchars=\\\{\}]
{\color{incolor}In [{\color{incolor}374}]:} \PY{c+c1}{\PYZsh{} CHDD Correlation matrix}
          \PY{n}{corrm} \PY{o}{=} \PY{n}{df}\PY{o}{.}\PY{n}{corr}\PY{p}{(}\PY{p}{)}
          \PY{n}{display}\PY{p}{(}\PY{n}{corrm}\PY{p}{)}
\end{Verbatim}


    
    \begin{verbatim}
               age       sex        cp  trestbps      chol       fbs  \
age       1.000000 -0.098447 -0.068653  0.279351  0.213678  0.121308   
sex      -0.098447  1.000000 -0.049353 -0.056769 -0.197912  0.045032   
cp       -0.068653 -0.049353  1.000000  0.047608 -0.076904  0.094444   
trestbps  0.279351 -0.056769  0.047608  1.000000  0.123174  0.177531   
chol      0.213678 -0.197912 -0.076904  0.123174  1.000000  0.013294   
fbs       0.121308  0.045032  0.094444  0.177531  0.013294  1.000000   
restecg  -0.116211 -0.058196  0.044421 -0.114103 -0.151040 -0.084189   
thalach  -0.398522 -0.044020  0.295762 -0.046698 -0.009940 -0.008567   
exang     0.096801  0.141664 -0.394280  0.067616  0.067023  0.025665   
oldpeak   0.210013  0.096093 -0.149230  0.193216  0.053952  0.005747   
slope    -0.168814 -0.030711  0.119717 -0.121475 -0.004038 -0.059894   
ca        0.276326  0.118261 -0.181053  0.101389  0.070511  0.137979   
thal      0.068001  0.210041 -0.161736  0.062210  0.098803 -0.032019   
target   -0.225439 -0.280937  0.433798 -0.144931 -0.085239 -0.028046   

           restecg   thalach     exang   oldpeak     slope        ca  \
age      -0.116211 -0.398522  0.096801  0.210013 -0.168814  0.276326   
sex      -0.058196 -0.044020  0.141664  0.096093 -0.030711  0.118261   
cp        0.044421  0.295762 -0.394280 -0.149230  0.119717 -0.181053   
trestbps -0.114103 -0.046698  0.067616  0.193216 -0.121475  0.101389   
chol     -0.151040 -0.009940  0.067023  0.053952 -0.004038  0.070511   
fbs      -0.084189 -0.008567  0.025665  0.005747 -0.059894  0.137979   
restecg   1.000000  0.044123 -0.070733 -0.058770  0.093045 -0.072042   
thalach   0.044123  1.000000 -0.378812 -0.344187  0.386784 -0.213177   
exang    -0.070733 -0.378812  1.000000  0.288223 -0.257748  0.115739   
oldpeak  -0.058770 -0.344187  0.288223  1.000000 -0.577537  0.222682   
slope     0.093045  0.386784 -0.257748 -0.577537  1.000000 -0.080155   
ca       -0.072042 -0.213177  0.115739  0.222682 -0.080155  1.000000   
thal     -0.011981 -0.096439  0.206754  0.210244 -0.104764  0.151832   
target    0.137230  0.421741 -0.436757 -0.430696  0.345877 -0.391724   

              thal    target  
age       0.068001 -0.225439  
sex       0.210041 -0.280937  
cp       -0.161736  0.433798  
trestbps  0.062210 -0.144931  
chol      0.098803 -0.085239  
fbs      -0.032019 -0.028046  
restecg  -0.011981  0.137230  
thalach  -0.096439  0.421741  
exang     0.206754 -0.436757  
oldpeak   0.210244 -0.430696  
slope    -0.104764  0.345877  
ca        0.151832 -0.391724  
thal      1.000000 -0.344029  
target   -0.344029  1.000000  
    \end{verbatim}

    
    The chart below is the Pearson's Correlation Coeficient matrix graphed
on a gradient. Purple is higher negative correlation, and green is the
higher positive correlation.

    \begin{Verbatim}[commandchars=\\\{\}]
{\color{incolor}In [{\color{incolor}375}]:} \PY{n}{plt}\PY{o}{.}\PY{n}{matshow}\PY{p}{(}\PY{n}{corrm}\PY{p}{)}\PY{p}{;}
          \PY{n}{plt}\PY{o}{.}\PY{n}{title}\PY{p}{(}\PY{l+s+s2}{\PYZdq{}}\PY{l+s+s2}{CHDD Correlation Matrix}\PY{l+s+s2}{\PYZdq{}}\PY{p}{)}\PY{p}{;}
\end{Verbatim}


    \begin{center}
    \adjustimage{max size={0.9\linewidth}{0.9\paperheight}}{output_25_0.png}
    \end{center}
    { \hspace*{\fill} \\}
    
    The correlation matrix gives us hints as wto which factors are
correlated with CAD. In particular is evident that Exercise Induced
Angina and Maximum HR are strongly correlated with the presence of CAD;
whereas Cholesterol and Fasting Blood Sugar are not strongly correlated
with CAD.

    \paragraph{Gender Analysis}\label{gender-analysis}

We grouped the dataset by gender to see if there were any gender
specific trends in the data.

    \begin{Verbatim}[commandchars=\\\{\}]
{\color{incolor}In [{\color{incolor}376}]:} \PY{c+c1}{\PYZsh{}grouped further by sex, 1 = male, 0 = female}
          
          \PY{n}{df\PYZus{}grouped} \PY{o}{=} \PY{n}{df}\PY{o}{.}\PY{n}{groupby}\PY{p}{(}\PY{p}{[}\PY{l+s+s1}{\PYZsq{}}\PY{l+s+s1}{target}\PY{l+s+s1}{\PYZsq{}}\PY{p}{,}\PY{l+s+s1}{\PYZsq{}}\PY{l+s+s1}{sex}\PY{l+s+s1}{\PYZsq{}}\PY{p}{]}\PY{p}{)}
          \PY{n+nb}{print} \PY{p}{(}\PY{n}{df\PYZus{}grouped}\PY{o}{.}\PY{n}{target}\PY{o}{.}\PY{n}{count}\PY{p}{(}\PY{p}{)}\PY{p}{)}\PY{p}{;}
          
          \PY{c+c1}{\PYZsh{}print (df\PYZus{}grouped.target.count())}
          \PY{n}{sns}\PY{o}{.}\PY{n}{countplot}\PY{p}{(}\PY{n}{x}\PY{o}{=}\PY{l+s+s1}{\PYZsq{}}\PY{l+s+s1}{sex}\PY{l+s+s1}{\PYZsq{}}\PY{p}{,}\PY{n}{hue}\PY{o}{=}\PY{l+s+s1}{\PYZsq{}}\PY{l+s+s1}{target}\PY{l+s+s1}{\PYZsq{}}\PY{p}{,}\PY{n}{data}\PY{o}{=}\PY{n}{df}\PY{p}{)}\PY{p}{;}
\end{Verbatim}


    \begin{Verbatim}[commandchars=\\\{\}]
target  sex
0       0       24
        1      114
1       0       72
        1       93
Name: target, dtype: int64

    \end{Verbatim}

    \begin{center}
    \adjustimage{max size={0.9\linewidth}{0.9\paperheight}}{output_28_1.png}
    \end{center}
    { \hspace*{\fill} \\}
    
    When grouped by gender and the presence of CAD.

\begin{itemize}
\tightlist
\item
  In the group of patients that do show CAD (target 1), the distribution
  is 93 males to 72 females, 1.3:1.
\item
  In the group that does not show CAD (target 0), the 114 males, 24
  females without the disease. , 4.75:1
\end{itemize}

Since the source population for both groups is the same, this
discrepancy alludes to a possible underlying factor which drives
physician's to order a coronary angiography for males when at a higher
rate than females event when the ultimate prognosis shows that the
patient did meet the criteria for a CAD diagnosis.

    \begin{Verbatim}[commandchars=\\\{\}]
{\color{incolor}In [{\color{incolor}377}]:} \PY{n}{groupdes} \PY{o}{=} \PY{n}{df\PYZus{}grouped}\PY{o}{.}\PY{n}{describe}\PY{p}{(}\PY{p}{)}
          \PY{n}{groupdes}
\end{Verbatim}


\begin{Verbatim}[commandchars=\\\{\}]
{\color{outcolor}Out[{\color{outcolor}377}]:}               age                                                        \textbackslash{}
                      count       mean        std   min    25\%   50\%    75\%   max   
          target sex                                                                
          0      0     24.0  59.041667   4.964913  43.0  56.75  60.5  62.00  66.0   
                 1    114.0  56.087719   8.385155  35.0  51.00  57.5  61.00  77.0   
          1      0     72.0  54.555556  10.265337  34.0  46.00  54.0  63.25  76.0   
                 1     93.0  50.903226   8.682897  29.0  44.00  52.0  57.00  70.0   
          
                         ca            {\ldots}   thalach        trestbps              \textbackslash{}
                      count      mean  {\ldots}       75\%    max    count        mean   
          target sex                   {\ldots}                                         
          0      0     24.0  1.291667  {\ldots}    154.75  174.0     24.0  146.125000   
                 1    114.0  1.140351  {\ldots}    156.00  195.0    114.0  131.929825   
          1      0     72.0  0.305556  {\ldots}    167.25  192.0     72.0  128.736111   
                 1     93.0  0.408602  {\ldots}    175.00  202.0     93.0  129.741935   
          
                                                                    
                            std    min    25\%    50\%    75\%    max  
          target sex                                                
          0      0    21.436078  108.0  130.0  140.0  152.5  200.0  
                 1    17.217361  100.0  120.0  130.0  140.0  192.0  
          1      0    16.536765   94.0  119.5  130.0  140.0  180.0  
                 1    15.955715   94.0  120.0  130.0  140.0  178.0  
          
          [4 rows x 96 columns]
\end{Verbatim}
            
    The average age of those diagnosed with CAD also varied by gender.

\begin{itemize}
\tightlist
\item
  54 for females\\
\item
  50 for males
\end{itemize}

Females referred for coronary angiography without CAD have an average
age of 59, which is higher than males at 56. However, the disparity in
group sample size skews the mean comparison. Liekwise for female
patients diagnosed with CAD, their average age was higher than males,
54,50.

    \paragraph{Resting Blood Pressure}\label{resting-blood-pressure}

We noted that the trestbps (resting blood pressure) is higher for people
without heart disease which seems to be contradictory.

The Cleveland Clinic
(https://health.clevelandclinic.org/6-myths-blood-pressure-heart-rate/)
defines Optimal blood pressure as 120 mm Hg systolic --- which is the
pressure as your heart beats --- over 80 mm Hg diastolic --- which is
the pressure as your heart relaxes."

Since our data only gives one number as opposed to the standard format
of systolic over diastolic, we will assume that it is the systolic
pressure. Given that assumption, all means of the target and non-target
groups are in the above normal level. The correlation matrix also showed
a weak correlation with resting BPs.

The dataset indicates that female study participants without CAD had the
highest BPS average, but this is ikely due to the fact that the highest
trestbps for the entire data set belongs to a woman in that category.
Since that group is so small, this outlying datapoint has skewed the
mean higher. The median of this group probably give a better estimation
of the true mean. The median for this group is 140 which is only a
little lower than the average and it is 10 points higher than all the
other medians so this is a trend that we need to further investigate.

    \paragraph{Maximum Heart Rate}\label{maximum-heart-rate}

In looking at thalach (maximum heart rate), there seems to be a very
wide spread. According to the Mayo Clinic
(https://www.mayoclinic.org/healthy-lifestyle/fitness/expert-answers/heart-rate/faq-20057979)
heart rate is influenced by MANY factors such as: age, fitness and
activity levels, smoker status, having cardiovascular disease, high
cholesterol or diabetes, air temperature, body position (standing up or
lying down, for example), emotions, body size, and medications. The
article also explains the wide spread we observe; "Although there's a
wide range of normal, an unusually high or low heart rate may indicate
an underlying problem. Consult your doctor if your resting heart rate is
consistently above 100 beats a minute (tachycardia) or if you're not a
trained athlete and your resting heart rate is below 60 beats a minute
(bradycardia)" According to the American heart association
(https://www.heart.org/en/healthy-living/fitness/fitness-basics/target-heart-rates),
for people age 50-54 (which is our average age) the Target HR Zone is
83-140 bpm and Average Maximum Heart Rate is 170 bpm. There are a few
data points that are above 170 and a considerable amount that are over
the 140 bpm.

    \subsubsection{Visualize Attributes}\label{visualize-attributes}

    We plotted the parameters with the highest absolute correlation, grouped
by sex to the Target and fit a linear model to these relationships to
get a sense of their individual contributions to the response.

    \begin{Verbatim}[commandchars=\\\{\}]
{\color{incolor}In [{\color{incolor}397}]:} \PY{c+c1}{\PYZsh{} Plot data and regression model fits across a FacetGrid.}
          \PY{n}{sns}\PY{o}{.}\PY{n}{lmplot}\PY{p}{(}\PY{n}{x}\PY{o}{=}\PY{l+s+s2}{\PYZdq{}}\PY{l+s+s2}{thalach}\PY{l+s+s2}{\PYZdq{}}\PY{p}{,} \PY{n}{y}\PY{o}{=}\PY{l+s+s2}{\PYZdq{}}\PY{l+s+s2}{target}\PY{l+s+s2}{\PYZdq{}}\PY{p}{,}\PY{n}{hue}\PY{o}{=}\PY{l+s+s2}{\PYZdq{}}\PY{l+s+s2}{sex}\PY{l+s+s2}{\PYZdq{}}\PY{p}{,} \PY{n}{data}\PY{o}{=}\PY{n}{df}\PY{p}{)}
          \PY{n}{sns}\PY{o}{.}\PY{n}{lmplot}\PY{p}{(}\PY{n}{x}\PY{o}{=}\PY{l+s+s2}{\PYZdq{}}\PY{l+s+s2}{cp}\PY{l+s+s2}{\PYZdq{}}\PY{p}{,} \PY{n}{y}\PY{o}{=}\PY{l+s+s2}{\PYZdq{}}\PY{l+s+s2}{target}\PY{l+s+s2}{\PYZdq{}} \PY{p}{,}\PY{n}{hue}\PY{o}{=}\PY{l+s+s2}{\PYZdq{}}\PY{l+s+s2}{sex}\PY{l+s+s2}{\PYZdq{}}\PY{p}{,}\PY{n}{data}\PY{o}{=}\PY{n}{df}\PY{p}{)}
          \PY{n}{sns}\PY{o}{.}\PY{n}{lmplot}\PY{p}{(}\PY{n}{x}\PY{o}{=}\PY{l+s+s2}{\PYZdq{}}\PY{l+s+s2}{chol}\PY{l+s+s2}{\PYZdq{}}\PY{p}{,} \PY{n}{y}\PY{o}{=}\PY{l+s+s2}{\PYZdq{}}\PY{l+s+s2}{target}\PY{l+s+s2}{\PYZdq{}} \PY{p}{,}\PY{n}{hue}\PY{o}{=}\PY{l+s+s2}{\PYZdq{}}\PY{l+s+s2}{sex}\PY{l+s+s2}{\PYZdq{}}\PY{p}{,}\PY{n}{data}\PY{o}{=}\PY{n}{df}\PY{p}{)}
          \PY{n}{sns}\PY{o}{.}\PY{n}{lmplot}\PY{p}{(}\PY{n}{x}\PY{o}{=}\PY{l+s+s2}{\PYZdq{}}\PY{l+s+s2}{oldpeak}\PY{l+s+s2}{\PYZdq{}}\PY{p}{,} \PY{n}{y}\PY{o}{=}\PY{l+s+s2}{\PYZdq{}}\PY{l+s+s2}{target}\PY{l+s+s2}{\PYZdq{}}\PY{p}{,}\PY{n}{hue}\PY{o}{=}\PY{l+s+s2}{\PYZdq{}}\PY{l+s+s2}{sex}\PY{l+s+s2}{\PYZdq{}}\PY{p}{,} \PY{n}{data}\PY{o}{=}\PY{n}{df}\PY{p}{)}
          \PY{n}{plt}\PY{o}{.}\PY{n}{show}\PY{p}{(}\PY{p}{)}\PY{p}{;}
\end{Verbatim}


    \begin{center}
    \adjustimage{max size={0.9\linewidth}{0.9\paperheight}}{output_36_0.png}
    \end{center}
    { \hspace*{\fill} \\}
    
    \begin{center}
    \adjustimage{max size={0.9\linewidth}{0.9\paperheight}}{output_36_1.png}
    \end{center}
    { \hspace*{\fill} \\}
    
    \begin{center}
    \adjustimage{max size={0.9\linewidth}{0.9\paperheight}}{output_36_2.png}
    \end{center}
    { \hspace*{\fill} \\}
    
    \begin{center}
    \adjustimage{max size={0.9\linewidth}{0.9\paperheight}}{output_36_3.png}
    \end{center}
    { \hspace*{\fill} \\}
    
    \begin{Verbatim}[commandchars=\\\{\}]
{\color{incolor}In [{\color{incolor}401}]:} \PY{n}{sns}\PY{o}{.}\PY{n}{lmplot}\PY{p}{(}\PY{n}{x}\PY{o}{=}\PY{l+s+s2}{\PYZdq{}}\PY{l+s+s2}{slope}\PY{l+s+s2}{\PYZdq{}}\PY{p}{,} \PY{n}{y}\PY{o}{=}\PY{l+s+s2}{\PYZdq{}}\PY{l+s+s2}{target}\PY{l+s+s2}{\PYZdq{}}\PY{p}{,}\PY{n}{hue}\PY{o}{=}\PY{l+s+s2}{\PYZdq{}}\PY{l+s+s2}{sex}\PY{l+s+s2}{\PYZdq{}}\PY{p}{,} \PY{n}{data}\PY{o}{=}\PY{n}{df}\PY{p}{)}\PY{p}{;}
          \PY{n}{sns}\PY{o}{.}\PY{n}{lmplot}\PY{p}{(}\PY{n}{x}\PY{o}{=}\PY{l+s+s2}{\PYZdq{}}\PY{l+s+s2}{ca}\PY{l+s+s2}{\PYZdq{}}\PY{p}{,} \PY{n}{y}\PY{o}{=}\PY{l+s+s2}{\PYZdq{}}\PY{l+s+s2}{target}\PY{l+s+s2}{\PYZdq{}}\PY{p}{,}\PY{n}{hue}\PY{o}{=}\PY{l+s+s2}{\PYZdq{}}\PY{l+s+s2}{sex}\PY{l+s+s2}{\PYZdq{}}\PY{p}{,} \PY{n}{data}\PY{o}{=}\PY{n}{df}\PY{p}{)}\PY{p}{;}
          \PY{n}{sns}\PY{o}{.}\PY{n}{lmplot}\PY{p}{(}\PY{n}{x}\PY{o}{=}\PY{l+s+s2}{\PYZdq{}}\PY{l+s+s2}{thal}\PY{l+s+s2}{\PYZdq{}}\PY{p}{,} \PY{n}{y}\PY{o}{=}\PY{l+s+s2}{\PYZdq{}}\PY{l+s+s2}{target}\PY{l+s+s2}{\PYZdq{}}\PY{p}{,}\PY{n}{hue}\PY{o}{=}\PY{l+s+s2}{\PYZdq{}}\PY{l+s+s2}{sex}\PY{l+s+s2}{\PYZdq{}}\PY{p}{,} \PY{n}{data}\PY{o}{=}\PY{n}{df}\PY{p}{)}\PY{p}{;}
\end{Verbatim}


    \begin{center}
    \adjustimage{max size={0.9\linewidth}{0.9\paperheight}}{output_37_0.png}
    \end{center}
    { \hspace*{\fill} \\}
    
    \begin{center}
    \adjustimage{max size={0.9\linewidth}{0.9\paperheight}}{output_37_1.png}
    \end{center}
    { \hspace*{\fill} \\}
    
    \begin{center}
    \adjustimage{max size={0.9\linewidth}{0.9\paperheight}}{output_37_2.png}
    \end{center}
    { \hspace*{\fill} \\}
    
    \subsubsection{Explore Joint Attributes and
Class}\label{explore-joint-attributes-and-class}

    \begin{Verbatim}[commandchars=\\\{\}]
{\color{incolor}In [{\color{incolor}403}]:} \PY{c+c1}{\PYZsh{}pull continuous numeric data}
          \PY{n}{continuous\PYZus{}features} \PY{o}{=} \PY{p}{[}\PY{l+s+s1}{\PYZsq{}}\PY{l+s+s1}{age}\PY{l+s+s1}{\PYZsq{}}\PY{p}{,} \PY{l+s+s1}{\PYZsq{}}\PY{l+s+s1}{trestbps}\PY{l+s+s1}{\PYZsq{}}\PY{p}{,} \PY{l+s+s1}{\PYZsq{}}\PY{l+s+s1}{chol}\PY{l+s+s1}{\PYZsq{}}\PY{p}{,} \PY{l+s+s1}{\PYZsq{}}\PY{l+s+s1}{thalach}\PY{l+s+s1}{\PYZsq{}}\PY{p}{,}\PY{l+s+s1}{\PYZsq{}}\PY{l+s+s1}{oldpeak}\PY{l+s+s1}{\PYZsq{}}\PY{p}{,} \PY{l+s+s1}{\PYZsq{}}\PY{l+s+s1}{target}\PY{l+s+s1}{\PYZsq{}}\PY{p}{]}
          
          \PY{c+c1}{\PYZsh{}changing to type float}
          \PY{n}{df}\PY{p}{[}\PY{n}{continuous\PYZus{}features}\PY{p}{]} \PY{o}{=} \PY{n}{df}\PY{p}{[}\PY{n}{continuous\PYZus{}features}\PY{p}{]}\PY{o}{.}\PY{n}{astype}\PY{p}{(}\PY{n}{np}\PY{o}{.}\PY{n}{float64}\PY{p}{)}
          
          \PY{c+c1}{\PYZsh{}matrix for continuous numeric data broken by heart disease presense, 1= yes 0=no}
          \PY{n}{sns}\PY{o}{.}\PY{n}{set}\PY{p}{(}\PY{n}{style}\PY{o}{=}\PY{l+s+s2}{\PYZdq{}}\PY{l+s+s2}{ticks}\PY{l+s+s2}{\PYZdq{}}\PY{p}{)}
          \PY{n}{grid} \PY{o}{=} \PY{n}{sns}\PY{o}{.}\PY{n}{pairplot}\PY{p}{(}\PY{n}{df}\PY{p}{,} \PY{n+nb}{vars}\PY{o}{=}\PY{p}{[}\PY{l+s+s1}{\PYZsq{}}\PY{l+s+s1}{age}\PY{l+s+s1}{\PYZsq{}}\PY{p}{,} \PY{l+s+s1}{\PYZsq{}}\PY{l+s+s1}{trestbps}\PY{l+s+s1}{\PYZsq{}}\PY{p}{,} \PY{l+s+s1}{\PYZsq{}}\PY{l+s+s1}{chol}\PY{l+s+s1}{\PYZsq{}}\PY{p}{,} \PY{l+s+s1}{\PYZsq{}}\PY{l+s+s1}{thalach}\PY{l+s+s1}{\PYZsq{}}\PY{p}{,}\PY{l+s+s1}{\PYZsq{}}\PY{l+s+s1}{oldpeak}\PY{l+s+s1}{\PYZsq{}}\PY{p}{]}\PY{p}{,} \PY{n}{hue}\PY{o}{=}\PY{l+s+s2}{\PYZdq{}}\PY{l+s+s2}{target}\PY{l+s+s2}{\PYZdq{}}\PY{p}{)}
          \PY{n}{grid} \PY{o}{=} \PY{n}{grid}\PY{o}{.}\PY{n}{map\PYZus{}lower}\PY{p}{(}\PY{n}{sns}\PY{o}{.}\PY{n}{kdeplot}\PY{p}{,} \PY{n}{cmap} \PY{o}{=} \PY{l+s+s1}{\PYZsq{}}\PY{l+s+s1}{Reds}\PY{l+s+s1}{\PYZsq{}}\PY{p}{)}\PY{p}{;}
\end{Verbatim}


    \begin{center}
    \adjustimage{max size={0.9\linewidth}{0.9\paperheight}}{output_39_0.png}
    \end{center}
    { \hspace*{\fill} \\}
    
    When our data is broken into two groups, consisting of those with
evidence of CAD and without we see overlapping in attributes and also
see noticeable differences. The matrix above contains the continuous
variables from our original 13 explanatory variables, we see three
different graphs including a correlation plot, density curve and
bivariate density.

The distributions show a steeper curve for those with heart disease.
This applies to variables old peak and thalach. This translates to those
with heart disease have a reversable blood disorder.

Those with heart disease reflect an age distribution at age 60, in
contrast the other group centers are age 75. Our matrix utilizing a
distribution and correlation plot did no show major differences between
male and females.

    \begin{Verbatim}[commandchars=\\\{\}]
{\color{incolor}In [{\color{incolor}385}]:} \PY{c+c1}{\PYZsh{}matrix broken down by gender, 1=male 0=female}
          
          \PY{n}{sex} \PY{o}{=} \PY{n}{sns}\PY{o}{.}\PY{n}{pairplot}\PY{p}{(}\PY{n}{df}\PY{p}{,} \PY{n+nb}{vars}\PY{o}{=}\PY{p}{[}\PY{l+s+s1}{\PYZsq{}}\PY{l+s+s1}{age}\PY{l+s+s1}{\PYZsq{}}\PY{p}{,} \PY{l+s+s1}{\PYZsq{}}\PY{l+s+s1}{trestbps}\PY{l+s+s1}{\PYZsq{}}\PY{p}{,} \PY{l+s+s1}{\PYZsq{}}\PY{l+s+s1}{chol}\PY{l+s+s1}{\PYZsq{}}\PY{p}{,} \PY{l+s+s1}{\PYZsq{}}\PY{l+s+s1}{thalach}\PY{l+s+s1}{\PYZsq{}}\PY{p}{,}\PY{l+s+s1}{\PYZsq{}}\PY{l+s+s1}{oldpeak}\PY{l+s+s1}{\PYZsq{}}\PY{p}{]}\PY{p}{,} \PY{n}{hue}\PY{o}{=}\PY{l+s+s2}{\PYZdq{}}\PY{l+s+s2}{sex}\PY{l+s+s2}{\PYZdq{}}\PY{p}{,} \PY{n}{palette}\PY{o}{=}\PY{l+s+s2}{\PYZdq{}}\PY{l+s+s2}{husl}\PY{l+s+s2}{\PYZdq{}}\PY{p}{)}
          \PY{k}{for} \PY{n}{i}\PY{p}{,} \PY{n}{j} \PY{o+ow}{in} \PY{n+nb}{zip}\PY{p}{(}\PY{o}{*}\PY{n}{np}\PY{o}{.}\PY{n}{triu\PYZus{}indices\PYZus{}from}\PY{p}{(}\PY{n}{sex}\PY{o}{.}\PY{n}{axes}\PY{p}{,} \PY{l+m+mi}{1}\PY{p}{)}\PY{p}{)}\PY{p}{:}
              \PY{n}{sex}\PY{o}{.}\PY{n}{axes}\PY{p}{[}\PY{n}{i}\PY{p}{,} \PY{n}{j}\PY{p}{]}\PY{o}{.}\PY{n}{set\PYZus{}visible}\PY{p}{(}\PY{k+kc}{False}\PY{p}{)}
\end{Verbatim}


    \begin{center}
    \adjustimage{max size={0.9\linewidth}{0.9\paperheight}}{output_41_0.png}
    \end{center}
    { \hspace*{\fill} \\}
    
    We breakdown the data by gender to see if there are noticeable
differences between males and females. While not much differences in age
and trestbps, there are substantial differences in chol and thalach.
Here we see that the high cholesterol point of 564 belongs to a female
as well as 4 other high points belong to females. But even with those
high values, overall females still have lower cholesterol levels than
males. The thalach seems to be substantially higher for females than
males. There also seems to be a negative correlation between thalach and
age.

    \subsubsection{New Features}\label{new-features}

    The data set used is originally from the UCI machine learning
repository. The kaggle version, which we used, is a simplified data set
with reduced explanatory and response variables. The original repository
has four different levels for our response, and not limited to heart
disease presence only, or a simple discrete binary value. Adding
additional response measure, we can get a better picture of what
contributes to the heart disease at different levels.

    \subsubsection{Exceptional Work}\label{exceptional-work}

    \begin{Verbatim}[commandchars=\\\{\}]
{\color{incolor}In [{\color{incolor}386}]:} \PY{c+c1}{\PYZsh{}splitting data into test and training sets, 30\PYZhy{}70 split}
          \PY{n}{X\PYZus{}train}\PY{p}{,} \PY{n}{X\PYZus{}test}\PY{p}{,} \PY{n}{y\PYZus{}train}\PY{p}{,} \PY{n}{y\PYZus{}test} \PY{o}{=} \PY{n}{train\PYZus{}test\PYZus{}split}\PY{p}{(}\PY{n}{df}\PY{o}{.}\PY{n}{drop}\PY{p}{(}\PY{l+s+s1}{\PYZsq{}}\PY{l+s+s1}{target}\PY{l+s+s1}{\PYZsq{}}\PY{p}{,}\PY{n}{axis}\PY{o}{=}\PY{l+m+mi}{1}\PY{p}{)}\PY{p}{,} 
                                                              \PY{n}{df}\PY{p}{[}\PY{l+s+s1}{\PYZsq{}}\PY{l+s+s1}{target}\PY{l+s+s1}{\PYZsq{}}\PY{p}{]}\PY{p}{,} \PY{n}{train\PYZus{}size}\PY{o}{=}\PY{o}{.}\PY{l+m+mi}{7}\PY{p}{,}\PY{n}{test\PYZus{}size}\PY{o}{=}\PY{l+m+mf}{0.30}\PY{p}{,} 
                                                              \PY{p}{)}
          \PY{c+c1}{\PYZsh{}train and predict}
          \PY{n}{logmodel} \PY{o}{=} \PY{n}{LogisticRegression}\PY{p}{(}\PY{n}{solver}\PY{o}{=}\PY{l+s+s1}{\PYZsq{}}\PY{l+s+s1}{liblinear}\PY{l+s+s1}{\PYZsq{}}\PY{p}{)}
          \PY{n}{logmodel}\PY{o}{.}\PY{n}{fit}\PY{p}{(}\PY{n}{X\PYZus{}train}\PY{p}{,}\PY{n}{y\PYZus{}train}\PY{p}{)}
          \PY{n}{predictions} \PY{o}{=} \PY{n}{logmodel}\PY{o}{.}\PY{n}{predict}\PY{p}{(}\PY{n}{X\PYZus{}test}\PY{p}{)}
          
          \PY{c+c1}{\PYZsh{}results}
          
          \PY{n+nb}{print}\PY{p}{(}\PY{n}{classification\PYZus{}report}\PY{p}{(}\PY{n}{y\PYZus{}test}\PY{p}{,}\PY{n}{predictions}\PY{p}{)}\PY{p}{)}
          
          \PY{c+c1}{\PYZsh{}confusion matrix}
          \PY{n}{CM} \PY{o}{=} \PY{n}{confusion\PYZus{}matrix}\PY{p}{(}\PY{n}{y\PYZus{}test}\PY{p}{,} \PY{n}{predictions}\PY{p}{)}
          \PY{n+nb}{print}\PY{p}{(}\PY{n}{CM}\PY{p}{)}
          
          \PY{n}{ax}\PY{o}{=} \PY{n}{plt}\PY{o}{.}\PY{n}{subplot}\PY{p}{(}\PY{p}{)}
          \PY{n}{sns}\PY{o}{.}\PY{n}{heatmap}\PY{p}{(}\PY{n}{CM}\PY{p}{,} \PY{n}{annot}\PY{o}{=}\PY{k+kc}{True}\PY{p}{,} \PY{n}{ax} \PY{o}{=} \PY{n}{ax}\PY{p}{)}\PY{p}{;} \PY{c+c1}{\PYZsh{}annot=True to annotate cells}
          
          \PY{c+c1}{\PYZsh{} labels, title and ticks}
          \PY{n}{ax}\PY{o}{.}\PY{n}{set\PYZus{}xlabel}\PY{p}{(}\PY{l+s+s1}{\PYZsq{}}\PY{l+s+s1}{Predicted labels}\PY{l+s+s1}{\PYZsq{}}\PY{p}{)}\PY{p}{;}\PY{n}{ax}\PY{o}{.}\PY{n}{set\PYZus{}ylabel}\PY{p}{(}\PY{l+s+s1}{\PYZsq{}}\PY{l+s+s1}{True labels}\PY{l+s+s1}{\PYZsq{}}\PY{p}{)}\PY{p}{;} 
          \PY{n}{ax}\PY{o}{.}\PY{n}{set\PYZus{}title}\PY{p}{(}\PY{l+s+s1}{\PYZsq{}}\PY{l+s+s1}{Confusion Matrix}\PY{l+s+s1}{\PYZsq{}}\PY{p}{)}\PY{p}{;} 
          \PY{n}{ax}\PY{o}{.}\PY{n}{xaxis}\PY{o}{.}\PY{n}{set\PYZus{}ticklabels}\PY{p}{(}\PY{p}{[}\PY{p}{]}\PY{p}{)}\PY{p}{;} \PY{n}{ax}\PY{o}{.}\PY{n}{yaxis}\PY{o}{.}\PY{n}{set\PYZus{}ticklabels}\PY{p}{(}\PY{p}{[}\PY{p}{]}\PY{p}{)}\PY{p}{;}
\end{Verbatim}


    \begin{Verbatim}[commandchars=\\\{\}]
             precision    recall  f1-score   support

        0.0       0.89      0.65      0.75        48
        1.0       0.70      0.91      0.79        43

avg / total       0.80      0.77      0.77        91

[[31 17]
 [ 4 39]]

    \end{Verbatim}

    \begin{center}
    \adjustimage{max size={0.9\linewidth}{0.9\paperheight}}{output_46_1.png}
    \end{center}
    { \hspace*{\fill} \\}
    
    Our response is binary implying a logistic regression model to help us
predict whether a patient has heart disease given the 13 explanatory
variables. We used 70-30 split for our training and test sets
respectively. The results are different when predicting those without
and with heart disease. Our accuracy is higher for predicting patients
without heart disease at the high 80\% range, ranging from 85-90\%.
Predicting heart disease however is lower range of 80s, ranging from 79
to 85\% during testing of several iterations.


    % Add a bibliography block to the postdoc
    
    
    
    \end{document}
